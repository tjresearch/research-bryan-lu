% Syslab Research Journal Template
% By Patrick White
% September 2019

% Do not edit this header
\documentclass[letterpaper,11pt]{article}
\usepackage{fullpage}
\usepackage{palatino}
\usepackage{enumitem}
\usepackage{courier}
\usepackage{graphicx}
\def\hrulefill{\leavevmode\leaders\hrule height 20pt\hfill\kern\z@}

% ------------- Edit these definitions ---------------------
\def\name{Bryan Lu}
\def\journalnum{15}
\def\daterange{02/10/20-02/17/20} % starts on Monday
\def\period{2}
% ------------------ END ---------------------------------
% Do not edit this
\begin{document}
	\thispagestyle{empty}
	\begin{flushright}
		{\Large Journal Report \journalnum} \\
		\daterange\\
		\name \\
		Computer Systems Research Lab \\
		Period \period, White
		\end{flushright}
	\hrule height 1pt

% ------ SECTION DAILY LOG -------------------------------------
\vspace{-0.8em}
\section*{Daily Log}
%Detail for each day about what you research, coded, debug, designed, created, etc. Informal style is OK.
\vspace{-0.8em}

\subsection*{Monday, February 10}
\vspace{-0.6em}
Isolated a sentence and a set of annotations by hand. Created and started debugging a new code segment to take a sentence, parse it for syntax, and process its annotations into a semantic tree. 
\vspace{-1.3em}
\subsection*{Tuesday, February 11}
\vspace{-0.6em}
Finished working out the debugging in the method taking annotations to semantic trees, corrected erroneous processing of annotation data. 
\vspace{-1.3em}
\subsection*{Thursday, February 13}
\vspace{-0.6em}
Watched presentations for part of class, and started to reformat my personal annotations from November/December into the same format. 
\vspace{-0.8em}

% ------ SECTION TIMELINE -------------------------------------
%\newpage
%\vspace{-1.7em}
\vspace{-1.0em}
\section*{Timeline}
\begin{tabular}{|p{1in}|p{2.5in}|p{2.5in}|}
	\hline
\textbf{Date} & \textbf{Goal} & \textbf{Met}\\ \hline 
	\hline
1/27 & Pending annotations, get everything up until the annotations step to work with olympiad problems, and work out other cosmetic details. & Yes! My snippet can output syntax parses for individual problems now, but I won't do local problem access until I have to run it multiple times with the tag model. \\
	\hline
2/3 & Assuming I have annotation data, run it through the parser to check it works, and format olympiad problems in that method. Else, begin trial and error process. & Couldn't do a lot of this because of presentations, but I will do this starting next week and into the week after. \\
	\hline
2/10 & Run annotation data combined with problems through code and turn them into semantic trees, reformat my annotated problems. & Finished (with a questionable level of success?), still working on reformatting my annotated olympiad problems. \\
	\hline
2/17 & Finalize annotation data for the olympiad problems and ensure that they produce valid, connected semantic trees. & N/A \\
	\hline 
2/24 & Start training the Naive Tag Model with olympiad problems, fix any issues that may arise. & N/A \\ 
	\hline 
\end{tabular}

\pagebreak
% ------ SECTION REFLECTION  ---------------------------------
\section*{Reflection}
%In narrative style, talk about your work this week. Successes, failures, changes to timeline, goals. This should also include concrete data, e.g. snippets of code, screenshots, output, analysis, graphs, etc.

The case I worked with this week as the subject of my annotation debugging was the following: 
\begin{center}
\textbf{sentence:} In the figure above, line AB, line CD, and line EF intersect at P. 

\textbf{annotations:} IsLine@5(line@6), IsLine@8(line@9), IsLine@12(line@13), CC@11(line@6, line@9), CC@11(line@6, line@13), IntersectAt@14(line@6, point@16)
\end{center}

After finishing debugging, the method outputs the following for the tree: 
\begin{center}
\textbf{tree:} [IsLine@5[line](\$AB:line@6[AB]), IsLine@8[line](\$CD:line@9[CD]), IsLine@12[line](\$EF:line@13[EF]), CC@11[and](\$AB:line@6[AB], \$CD:line@9[CD]), CC@11[and](\$AB:line@6[AB], \$EF:line@13[EF]), IntersectAt@14[intersect](\$AB:line@6[AB], \$P:point@16[P])]
\end{center}
This formatting is sort of weird, which may be due to trying to turn the Semantic Tree Nodes that are supposed to be returned into strings, but I think the method does the job. Each element (a relation) is an edge between two objects, each of which has a type and a name. Because of this, I think my olympiad problems may have issues when referring to more abstract objects or objects that are defined based on others (i.e. the circumcircle of $ABC$, for instance), but this might be okay. We'll see how this goes in the next few weeks!
\end{document}

