% Syslab Research Journal Template
% By Patrick White
% September 2019

% Do not edit this header
\documentclass[letterpaper,11pt]{article}
\usepackage{fullpage}
\usepackage{palatino}
\usepackage{enumitem}
\usepackage{courier}
\usepackage{graphicx}
\def\hrulefill{\leavevmode\leaders\hrule height 20pt\hfill\kern\z@}

% ------------- Edit these definitions ---------------------
\def\name{Bryan Lu}
\def\journalnum{8}
\def\daterange{10/28/19-11/11/19} % starts on Monday
\def\period{2}
% ------------------ END ---------------------------------
% Do not edit this
\begin{document}
	\thispagestyle{empty}
	\begin{flushright}
		{\Large Journal Report \journalnum} \\
		\daterange\\
		\name \\
		Computer Systems Research Lab \\
		Period \period, White
		\end{flushright}
	\hrule height 1pt

% ------ SECTION DAILY LOG -------------------------------------
\vspace{-0.8em}
\section*{Daily Log}
%Detail for each day about what you research, coded, debug, designed, created, etc. Informal style is OK.
\vspace{-0.8em}

\subsection*{Monday, October 28}
\vspace{-0.6em}
Began sketching out what different cases of objects should return for the dependency tree distance method, and sketched out implementations. 
\vspace{-1.3em}
\subsection*{Tuesday, October 29}
\vspace{-0.6em}
I coded the method, neglecting the sign of the output (will be taken to be positive whenever possible). 
\vspace{-1.3em}
\subsection*{Thursday, October 31}
\vspace{-0.6em}
Presentations. 
\vspace{-1.3em}
\subsection*{Wednesday, November 7}
\vspace{-0.6em}
I finished the dependency tree distance method and began researching the \texttt{scikit} library and the logistic learning method.
\vspace{-1.3em}
\subsection*{Thursday, November 8}
\vspace{-0.6em}
I realized I had a shortage of triangular geometry test cases (the easiest possible cases to train on), and I started assembling test cases.
% ------ SECTION TIMELINE -------------------------------------
%\newpage
%\vspace{-1.7em}
\vspace{-1.5em}
\section*{Timeline}
\begin{tabular}{|p{1in}|p{2.5in}|p{2.5in}|}
	\hline
\textbf{Date} & \textbf{Goal} & \textbf{Met}\\ \hline 
	\hline 
10/21 & Complete the extraction of all the necessary features of the sentences needed as input data. & Almost -- I have found all of the necessary features except for ``dependency tree distance.'' \\
	\hline 
10/28 & Begin writing code to create a log-linear classifier using \texttt{scikit}, and finalize the inputs needed for the algorithm. & Yes, I have all of the features I need per sentence.   \\
	\hline
11/4 & Create the log-linear classifier/learning algorithm and the training data, and begin testing. & I've started, but I've realized this is a rather ambitious task because I still need to finish creating my training data.\\
	\hline
11/11 & Complete a set of about 40 problems to serve as my training data set, with the correct relations. & N/A \\
	\hline
11/18 & Begin testing, tweaking previous steps as needed and adding more cases if needed. & N/A \\
	\hline
\end{tabular}


% ------ SECTION REFLECTION  ---------------------------------
\section*{Reflection}
%In narrative style, talk about your work this week. Successes, failures, changes to timeline, goals. This should also include concrete data, e.g. snippets of code, screenshots, output, analysis, graphs, etc.

The past two weeks have been fairly interrupted, with presentations and the end of the quarter, but I tried to keep progressing as much as I could. I may have fallen a week or two behind schedule at this point, but I hope I can make up the time later. 

I faced a decent amount of uncertainty when coding the dependency tree distance method. In the end, I decided to enact the following scheme that seemed to be fairly reasonable, restricting the distances to reals between 0 and 3: 
\begin{enumerate}[label=\alph*)]
\item If both of the objects passed are explicit, i.e. $ABC$ (a triangle) or $AB$ (a segment), then report the difference in length between the strings, scaled arbitrarily by 0.1.
\item If one of the objects is explicit and the other is an abstract concept, i.e. $ABC$ vs. ``circle,'', I gave this a distance of 3 by default. 
\item If both are abstract concepts, check if both represent ``closed'' shapes, i.e. circles, triangles, polygons. 
\begin{enumerate}[label=\roman*)]
\item If only one does, return a distance of 2. 
\item Else, check how many relations/objects that are both associated with them. Return 1.5 minus the amount of overlapping relations/objects times 0.1.
\end{enumerate}
\end{enumerate}
I'm not sure how well this will work, but I can always adjust it later. 

I've also realized I need more test cases for training, especially with regards to triangular geometry problems like the following, for which it's really easy to create the correct relations for: 
\begin{center}
\textit{Let Gamma be the circumcircle of acute triangle ABC. Points D and E are on segments AB and AC respectively
such that AD = AE. The perpendicular bisectors of BD and CE intersect minor arcs AB and AC of Gamma at
points F and G respectively. Prove that lines DE and FG are either parallel or they are the same line.}
\end{center}
I think I will need to select a decent number of these to use for training my model, which I will do in the next week. 
\end{document}

