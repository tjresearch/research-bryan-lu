% Syslab Research Journal Template
% By Patrick White
% September 2019

% Do not edit this header
\documentclass[letterpaper,11pt]{article}
\usepackage{fullpage}
\usepackage{palatino}
\usepackage{enumitem}
\usepackage{courier}
\usepackage{graphicx}
\def\hrulefill{\leavevmode\leaders\hrule height 20pt\hfill\kern\z@}

% ------------- Edit these definitions ---------------------
\def\name{Bryan Lu}
\def\journalnum{9}
\def\daterange{11/11/19-11/18/19} % starts on Monday
\def\period{2}
% ------------------ END ---------------------------------
% Do not edit this
\begin{document}
	\thispagestyle{empty}
	\begin{flushright}
		{\Large Journal Report \journalnum} \\
		\daterange\\
		\name \\
		Computer Systems Research Lab \\
		Period \period, White
		\end{flushright}
	\hrule height 1pt

% ------ SECTION DAILY LOG -------------------------------------
\vspace{-0.8em}
\section*{Daily Log}
%Detail for each day about what you research, coded, debug, designed, created, etc. Informal style is OK.
\vspace{-0.8em}

\subsection*{Monday, November 11}
\vspace{-0.6em}
Extracted problems from \texttt{https://imogeometry.blogspot.com}, a curated website for all olympiad geometry problems on AoPS in the Contest Collections, and performed some string parsing to get rid of unicode characters. 
\vspace{-1.3em}
\subsection*{Tuesday, November 12}
\vspace{-0.6em}
Randomly generated problems from China, Turkey, and Iranian Team Selection Tests (TSTs), and from the ELMO (a MOP contest), and picked 40 triangle geometry problems from this set of problems. 

\vspace{-1.3em}
\subsection*{Thursday, November 14}
\vspace{-0.6em}
Began assigning all of the valid relations associated with the  the selected problems on Tuesday, and adjusted \texttt{lexicon.txt} and \texttt{finalrelations.txt} to accomodate new relations I hadn't anticipated. 

\vspace{-1.0em}

% ------ SECTION TIMELINE -------------------------------------
%\newpage
%\vspace{-1.7em}
\vspace{-1.0em}
\section*{Timeline}
\begin{tabular}{|p{1in}|p{2.5in}|p{2.5in}|}
	\hline
\textbf{Date} & \textbf{Goal} & \textbf{Met}\\ \hline 
	\hline 
10/28 & Begin writing code to create a log-linear classifier using \texttt{scikit}, and finalize the inputs needed for the algorithm. & Yes, I have all of the features I need per sentence.   \\
	\hline
11/4 & Create the log-linear classifier/learning algorithm and the training data, and begin testing. & I've started, but I've realized this is a rather ambitious task because I still need to finish creating my training data.\\
	\hline
11/11 & Complete a set of about 40 problems to serve as my training data set, with the correct relations. & I have the problems, but putting in the correct relations for these problems is a lot of work. \\
	\hline
11/18 & Build the model with \texttt{scikit}, tweaking previous steps as needed, and finish the necessary test cases. & N/A \\
	\hline
11/25 & Test and train the logistic model, and see if any methods could be added to improve accuracy. & N/A \\
	\hline 
Winter Goal & Be able to output a set of possible literals (statements) based on detected relations in the problem. & N/A \\
	\hline 
\end{tabular}


% ------ SECTION REFLECTION  ---------------------------------
\section*{Reflection}
%In narrative style, talk about your work this week. Successes, failures, changes to timeline, goals. This should also include concrete data, e.g. snippets of code, screenshots, output, analysis, graphs, etc.

This week, my lack of test cases from first quarter started to really impact my work this week. Luckily, I sent out a forum post to the community a while ago, asking if anyone knew of another compendium of olympiad geometry problems. Sure enough, the community pulled through and some user on AoPS has been doing exactly that! I took the liberty of ripping some of the problems from his blog (but I'm not sure how I should eventually cite this source...) 

This still doesn't account for the fact that I still have to create all of my test cases myself -- each problem takes a decent amount of time to complete. Here's one of the problems that I'll be testing with, with the correct relations I have associated with the problem: 

\begin{center}
\textbf{Problem: }Let Gamma be the circumcircle of acute triangle ABC. Points D and E are on segments AB and AC respectively
such that AD = AE. The perpendicular bisectors of BD and CE intersect minor arcs AB and AC of Gamma at
points F and G respectively. Prove that lines DE and FG are either parallel or they are the same line.
\end{center}

\begin{center}
\textbf{Relations: }
IsTriangle(ABC) \quad 
IsCircle(Gamma) \quad 
IsCircumcircle(Gamma, ABC) \quad 
IsSegment(AB) \quad 
IsSegment(AC) \quad 
IsPoint(D) \quad 
IsPoint(E) \quad 
Equals(AD, AE) \quad \\
Collinear(D, A, B) \quad 
Collinear(E, A, C) \quad 
IsLine(DE) \quad 
IsLine(FG) \quad 
IsParallel(DE, FG) \quad 
\end{center}

This will probably expand to include more information about the perpendicular bisectors being accepted as well. 

I'm fairly sure that I can finish these test cases this week, but considering the volume of tests associated with each problem, I might not need all of these to train the model. After all, I need to consider all possible pairs of explicit/abstract variables when testing, as long as they're close enough in the statement/in the same sentence, which is still a decent number of pairs per problem.

I am a bit concerned with my progress towards my winter goal -- I think it's reasonable, but if I encounter another major challenge to my process in the building and testing of the logistic model I may not be able finish. Fingers crossed that everything goes smoothly. 
\end{document}

