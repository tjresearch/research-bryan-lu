% Syslab Research Journal Template
% By Patrick White
% September 2019

% Do not edit this header
\documentclass[letterpaper,11pt]{article}
\usepackage{fullpage}
\usepackage{palatino}
\usepackage{enumitem}
\usepackage{courier}
\usepackage{graphicx}
\def\hrulefill{\leavevmode\leaders\hrule height 20pt\hfill\kern\z@}

% ------------- Edit these definitions ---------------------
\def\name{Bryan Lu}
\def\journalnum{A}
\def\daterange{04/13/20-04/20/20} % starts on Monday
\def\period{2}
% ------------------ END ---------------------------------
% Do not edit this
\begin{document}
	\thispagestyle{empty}
	\begin{flushright}
		{\Large Journal Report \journalnum} \\
		\daterange\\
		\name \\
		Computer Systems Research Lab \\
		Period \period, White
		\end{flushright}
	\hrule height 1pt

% ------ SECTION DAILY LOG -------------------------------------
\vspace{-0.8em}
\section*{Weekly Updates}
%Detail for each day about what you research, coded, debug, designed, created, etc. Informal style is OK.
Hi Dr. White! Just checking in and hope you're doing okay during these weird times.

I didn't do much this week, but I have been taking some time to do some learning on my own! I started to crack into an intermediate mechanics textbook (Goldstein) and it's challenging, but I'm trying to get as much as I can out of it. I've also been keeping my geometry skills sharp, as I've taken some time recently to curate an olympiad geometry problem set for VMT. I think I may end up using some of them as test cases for my project down the line, but I may not. 

The one thing that I don't really want to handle right now in regards to my project is to figure out input/output stuff pertaining to getting code to read the test cases and process them -- I'll try to get on it next week because it's not hard, but I find it really annoying to do. 

Finally, a miscellaneous problem that I've done recently (and I think you might appreciate working on it):  \\

\noindent \textbf{(2020 Spring OMO \# 18, roughly paraphrased)} Vincent has a fair six-sided die numbered from 1 to 6. At time $t = 1$s, he rolls the die and records it on a piece of paper. Every second thereafter, he rolls the die again. If the die's result does \textbf{not} match the result of the previous roll, Vincent records the most recent value on the sheet of paper and gets to keep rolling the die. Otherwise, Vincent stops rolling and has to find the (arithmetic) mean of all of the numbers currently on his piece of paper, $M$. Given that Vincent rolled a 1 at $t = 1$s, 
find $E[M]$. (Your answer should be of the form $r - s \ln t$, where $r, s, t$ are positive rational numbers.)\\

Hope we can actually contact each other at some point this week that's not over email!


\end{document}

